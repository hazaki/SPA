\chapter{Authenticité et Intégrité}
Après avoir présenté le procédé, nous allons détailler davantage les outils utilisés pour le satisfaire.

Pour palier à la faiblesse du \emph{Port Knocking}, et afin qu'une simple écoute du réseau ne suffise pas à mettre en péril ce schéma, la pérénnité de cette méthode va reposer sur l'authentification des clients. En effet il est nécessaire de vérifier que l'acceptation d'un paquet par le pare-feu est bien effectuée par un client légitime.

A contrario du \emph{Port Knocking} qui supposait que seul un client autorisé pouvait connaître la séquence de coups, ici l'authenticité va se faire grâce à des clés prépartagées. En effet, le serveur va partager une clé avec chacun des ses clients, chaque clé étant différente des autres. 
Le client génère alors un paquet SPA chiffré avec cette clé, le serveur SPA le déchiffre avec la même clé. Il détermine la légitimité de la requête et si les champs sont valides, l'intégrité grâce au système de hachage

Si la demande est valide, le client sera considéré comme authentifié auprès du serveur et il pourra communiquer avec lui sur le port fourni.

\section{Chiffrement}

Pour le chiffrement, nous utiliserons le standard du chiffrement symétrique, AES. 
Pour choisir le mode d'opération cryptographique, nous avons du choisir entre plusieurs modes. 
Sachant que nos données à chiffrer sont de taille limitée, nous n'utiliserons pas un mode de chiffrement qui se comporte comme un chiffrement par flux.
De plus, nous savons que ECB connait des vulnérabilités quant à l'intégrité et la protection des données puisqu'il est sensible aux attaques par répétition (deux blocs avec le même contenu seront chiffrés de la même manière).
%Par ailleurs, OFB est fragile vis-à-vis des attaques à clair connu et CTR est fragile si l'attaquant connait l'IV (ce qui sera le cas ici étant donné que celui-ci sera envoyé en clair).
Nous choisirons donc CBC.


Nous désirons que le client puisse choisir le port serveur avec lequel il veut communiquer, le protocole de transport utilisé (tcp ou udp) ainsi que le temps d'accès au port (en maintenant un temps maximal d'ouverture de port).

\section{Champ \textbf{\emph{IP}} du paylaod}

De plus, afin de s'assurer de l'authenticité de l'émetteur du paquet, celui-ci doit introduire son adresse IP dans le champ de données chiffré. Ainsi, en comparant cette donnée et le header du paquet ip, il sera possible de voir si les informations sont cohérentes et donc authentiques (un attaquant n'étant pas en mesure de fournir un chiffré valide sans la clé adéquate).

Un utilisateur légitime construit naturellement un paquet avec ces deux champs IP identiques. Un usurpateur pourra alors être démasqué si sa requête ne respecte pas cela.

\section{Hachage}
\begin{quotation}
\emph{"Une fonction de hachage [...] doit être rapide à calculer, transforme un message de longueur arbitraire en une empreinte numérique de taille fixe. Cette dernière est ensuite signée ..."}
\end{quotation}
\underline{\emph{Codage, cryptologie et applications}}, \textbf{Bruno Martin}\\

l'intérêt des fonctions de hachage est de vérifier l'intégrité du champ de donnée et nous rendons le haché authentique grâce au chiffrement par AES.
Il nous faut maintenant déterminer quel algorithme est le plus adapté sachant que les plus courant sont MD5, SHA1, SHA-256 ou SHA-512.\newline

Le système MD5, était considéré comme sûr avant que les chercheurs chinois, \emph{Xiaoyun Wang, Dengguo Feng, Xuejia Lai (co-inventeur de IDEA) et Hongbo Yu "} ne soient capables de trouver des collisions en quelques heures et le brisent.
le système SHA-1 est, lui aussi, sensible aux collisions notamment à celles basées sur le paradoxe des anniversaires.
Nous avons donc choisi SHA-256 plutôt que SHA-512 pour la taille restreinte du haché produit tout en sachant que le réel intérêt de ce haché est d'établir une simple somme de contrôle.\newline

Nous hachons tous les champs de données de la requête et la concaténons à celle-ci puis le tout sera chiffré afin de préserver l'intégrité des données. Si le chiffré est changé, le haché ne correspondra plus aux champs de donnée.\\

Nous avons mis en place un système permettant d'authentifier les requêtes arrivant au serveur SPA ainsi que vérifier l'intégrité de celles-ci, nous allons maintenant nous intéresser à contrer certains attaques supplémentaires.
 %http://fr.wikipedia.org/wiki/MD5
 
 
