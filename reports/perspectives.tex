\chapter{Perspectives d'évolution}

Nous avons vu que notre implémentation satisfaisait un fonctionnement de base du principe de SPA.
Cependant, nous avons noté quelques perspectives d'amélioration qui rendraient le tout plus robuste et complet.

\section{Mise à jour des OTP}
Dans une précédente partie, nous avons détaillé le choix de notre système de gestion des clefs a usages unique. 
Celles-ci sont générées par une \emph{seed} et un compteur qui évolue à chaque envoie d'un paquet SPA.
Nous faisons actuellement évoluer ce compteur de façon analogue pour chaque client.
Ainsi, la connaissance à un moment donné de la \emph{seed} et du compteur pourrait permettre à n'importe quel autre partie ayant conscience du mécanisme d'évolution du compteur de déterminer la clef suivante.

\section{Appel a la librairie Libnetfilter}
En ce qui concerne l'intéraction concrète avec les règles iptables du pare-feu, nous utilisons des appels systèmes simples. Cependant, l'utilisation de librairies plus spécifiques telles que \emph{libnetfilter} pourrait sembler plus adapté.