\chapter*{Introduction}

Afin d'empêcher le scan de port d'un pare-feu protégeant un serveur applicatif, de nombreuses techniques ont vu le jour. En effet, posséder un pare-feu dont tous les ports sont fermés et ne s'ouvrent que lors d'échanges avec des clients approuvés permettrait d'amoindrir les attaques sur ce serveur.

Le but est donc de fermer par défaut les ports logiciels du pare-feu et mettre en place un mécanisme de reconnaissance des clients autorisés à interagir avec le serveur.

De plus, il existe une restriction supplémentaire qui est que le client doit pouvoir faire ouvrir un port au serveur via un unique paquet envoyé.

Plusieurs modèles ont tentés de répondre à cette question, nous allons étudier les plus connus.
