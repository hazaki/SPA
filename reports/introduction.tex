\chapter*{Introduction}

Le Single Packet Authorization est une technique de communication sécurisée qui permet de vérifier l'authenticité de tous clients désirant accéder à un service.

Afin de permettre à un serveur applicatif de ne communiquer qu'avec des clients désirés, de nombreuses techniques ont vu le jour. 

En effet, posséder un pare-feu dont tous les ports sont fermés mais qui, pour une période donnée, permettrait des échanges avec des clients approuvés permettrait d'amoindrir les risques d'attaques sur ce serveur.

Le but est de fermer tous les ports d'un pare-feu par défaut et de mettre en place un mécanisme de reconnaissance des clients autorisés à interagir avec le serveur qui auraient la possibilité de demander une ouverture de port.

De plus, il existe une restriction supplémentaire qui est que le client doit pouvoir demander une ouverture de port au pare-feu via un unique paquet envoyé.

Plusieurs modèles ont tentés de répondre à cette question, nous allons étudier les plus connus.
