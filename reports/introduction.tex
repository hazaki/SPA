\chapter*{Introduction}

Dans le cadre de la sécurisation de communication client-serveur, il peut être intéressant d'avoir un pare-feu côté serveur dont les règles sont modifiées dynamiquement afin que toute personne ne soit pas capable de communiquer avec, sans être authentifié.

Dans ce contexte, un client légitime demanderait une autorisation de communication avec le serveur qui se traduirait par l'ajout de règles permissives sur le pare-feu.

Cette méthode nécessite d'authentifier ces clients afin de leur accorder des privilèges particuliers (accéder à un port donné du serveur applicatif pour une période restreinte).

Cette technique est une sécurité supplémentaire que l'on ajoute à celles déjà connues, elle n'a pas vocation à remplacer d'autres méthodes mais à s'y ajouter. Ainsi, si la sécurité de cette méthode est malmenée, l'attaquant se retrouve face à une architecture réseau sécurisée.

Pour répondre à cette problématique, deux systèmes ont été imaginés.

Le premier, le \emph{Port Knocking}, consiste  à laisser l'accès à un port du serveur applicatif à tous clients connaissant une séquence de tentative de connexion particulière et nécessaire à son accès. Cette séquence représente le secret partagé entre le serveur et le client.

Le secret du second mécanisme, le \emph{SPA} repose sur la présence de clés servant à un chiffrement symétrique.
Dans ce cadre, la vérification de l'authenticité et de l'intégrité  de la requête se fait grâce à ce système de clé alors que l'authenticité du \emph{Port Knocking} repose sur la connaissance d'une séquence adéquate. 

Le but de cette démarche de recherche et d'implémentation va donc être d'apporter une solution respectant l'intégrité et l'authenticité en se basant sur le modèle SPA (la fragilité du \emph{Port Knoking} fasse au attaques en Man In The Middle le rendant trop fragile). 

Le principe général du SPA consiste à interdire toute communication vers le serveur applicatif avec le pare-feu et de mettre en place un mécanisme de reconnaissance des clients autorisés à interagir avec le serveur qui auraient la possibilité de demander l'ajout de règles permissives sur le pare-feu.

De plus, le client ne peut générer qu'un seul paquet pour formuler sa demande (principe du \emph{Single Packet Authorization}).

Après une introduction de ces sytèmes, nous nous proposons de détailler le principe du SPA, sa résistance aux attaques (rejeu et DoS) ainsi que nos choix d'implémentation avant de proposer des tests.
