\chapter{Single Packet Authorization}

Le Single Packet Authorization est le second modèle qui répond à cette problématique.

Ce Protocole fonctionne grâce à l'utilisation d'un chiffrement symétrique. En effet, pour demander une ouverture de port, le client va envoyer le numéro de port chiffré avec un clé partagée avec le serveur. Le serveur aura une clé différente par client ainsi, aucune identité ne pourra être usurpée.

Grâce à des principes supplémentaires expliqués ultérieurement, il sera possible d'empêcher des attaques de type rejeu.

Nous nous proposons d'étudier plus en profondeur cette méthode.

%\includegraphics[scale=0.5]{schema_general_spa}
