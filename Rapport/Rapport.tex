\documentclass[12pt]{article}
\usepackage[utf8]{inputenc}
\usepackage[T1]{fontenc}
\usepackage[francais]{babel}
\usepackage{eso-pic,graphicx}
\usepackage{xcolor}
\usepackage{url}
\usepackage{hyperref, cite}
\usepackage{amsmath,amsfonts, amsthm,amssymb}
\usepackage{algorithmicx}
\usepackage{algpseudocode}
\usepackage{vmargin}
\usepackage{multirow}
\usepackage{listings}
\title{Rapport de projet:\\
Mécanismes de Single Packet Authorization}
\author{CEOLA Cédric MONIN Jacques\\
Encadré par M. GUERMOUCHE Abdou}

\begin{document}

\maketitle

\clearpage                  
\tableofcontents
\clearpage

\section{Introduction}

Afin d'empêcher le scan de port d'un pare-feu protégeant un serveur applicatif, de nombreuses techniques ont vu le jour. En effet, posséder un pare-feu dont tous les ports sont fermés et ne s'ouvrent que lors d'échanges avec des clients approuvés permettrait d'amoindrir les attaques sur ce serveur.

Le but est donc de fermer par défaut les ports logiciels du pare-feu et mettre en place un mécanisme de reconnaissance des clients autorisés à interagir avec le serveur.

Plusieurs modèles ont tentés de répondre à cette question, nous allons étudier les plus connus.

\subsection{Port Knocking}

Le Port Knocking est un des modèles qui a été inventé pour répondre au besoin d'ouverture et de fermeture automatique de ports.

Dans ce modèle, la client s'authentifie auprès du serveur en lançant des tentatives de connexions sur certains ports spécifiques constituant un code. Le serveur ouvre un port seulement si le même émetteur réalise la bonne séquence de tentative de connexion.
Cette solution nécessite peu de ressource et est aisée à mettre en place, cependant une simple écoute du réseau permet de repérer le code permettant l'ouverture d'un port ce qui rend cette solution facilement attaquable.

\subsection{Single Packet Authorization}

\clearpage

\section{Principes fondamentaux du SPA}
 
\clearpage 
 
\section{Notre Implémentation} 
 
\subsection{Choix de Librairie}

http://packetlife.net/armory/category/encryption/

\subsubsection{chiffrer}
$http://en.wikipedia.org/wiki/AES_implementations
http://en.wikipedia.org/wiki/CyaSSL
http://en.wikipedia.org/wiki/GnuTLS
http://en.wikipedia.org/wiki/PolarSSL
http://en.wikipedia.org/wiki/OpenSSL$

\subsubsection{forger des paquets}
$libdnet ?
libnet$

\subsubsection{capturer des paquets}
$libpcap$

\end{document}
