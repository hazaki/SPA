\documentclass[12pt]{article}
\usepackage[utf8]{inputenc}
\usepackage[T1]{fontenc}
\usepackage[francais]{babel}
\usepackage{eso-pic,graphicx}
\usepackage{xcolor}
\usepackage{url}
\usepackage{hyperref, cite}
\usepackage{amsmath,amsfonts, amsthm,amssymb}
\usepackage{algorithmicx}
\usepackage{algpseudocode}
\usepackage{vmargin}
\usepackage{multirow}
\usepackage{listings}
\title{Rapport de projet:\\
Mécanismes de Single Packet Authorization}
\author{CEOLA Cédric MONIN JAcques\\
Encadré par M. GUERMOUCHE Abdou}

\begin{document}

\maketitle

\clearpage                  
\tableofcontents
\clearpage

\section{Introduction}
<<<<<<< HEAD

Dans le cadre de l'UE de projet, nous allons implémenter un système de \emph{Single Packet AUtorization}.\\
m\\
m\\
m\\
m\\
=======
Dans le cadre de l'UE de projet, nous allons implémenter un système de \emph{Single Packet AUtorization}.

>>>>>>> c1fd67caef5c788782a9c18ec8869aa840607133
\section{Choix de Librairie}

http://packetlife.net/armory/category/encryption/

\subsection{chiffrer}
http://en.wikipedia.org/wiki/AES_implementations
http://en.wikipedia.org/wiki/CyaSSL
http://en.wikipedia.org/wiki/GnuTLS
http://en.wikipedia.org/wiki/PolarSSL
http://en.wikipedia.org/wiki/OpenSSL

\subsection{forger des paquets}
libdnet ?
libnet

\subsection{capturer des paquets}
libpcap

\end{document}
